\documentclass[12pt]{IEEEtran}%
\usepackage{amsmath}
\usepackage{cite}
\usepackage{graphicx}
\usepackage{graphics}
\usepackage{geometry}
\usepackage{setspace}
\usepackage{amsfonts}
\usepackage{amssymb}
\usepackage{hyperref}
\usepackage{comment}
\usepackage{algorithm}
\usepackage{algpseudocode}%
\setcounter{MaxMatrixCols}{30}
%TCIDATA{OutputFilter=latex2.dll}
%TCIDATA{Version=5.50.0.2953}
%TCIDATA{CSTFile=IEEEtran.cst}
%TCIDATA{Created=Monday, February 05, 2001 16:11:19}
%TCIDATA{LastRevised=Saturday, September 29, 2018 22:28:54}
%TCIDATA{<META NAME="GraphicsSave" CONTENT="32">}
%TCIDATA{<META NAME="SaveForMode" CONTENT="1">}
%TCIDATA{BibliographyScheme=BibTeX}
%TCIDATA{Language=American English}
%BeginMSIPreambleData
\providecommand{\U}[1]{\protect\rule{.1in}{.1in}}
%EndMSIPreambleData
\hypersetup{
    colorlinks=true,
    linkcolor=black,
    filecolor=magenta,      
    urlcolor=black,
    citecolor=black
}
\geometry{left=1in,right=1in,top=1in,bottom=1in}
\begin{document}
%
%TCIMACRO{\QSubDoc{Include IEEEHeading}{%TCIDATA{LaTeXparent=0,0,RUL1.tex}

\title{A Neural Network-Evolutionary Computational Framework for Remaining Useful Life Estimation of Aircraft Engine}

\author{David Laredo\thanks{David Laredo, Zhaoyin Chen and J. Q. Sun are with Department of Mechanical Engineering, University of California, Merced, CA 95343.}, Zhaoyin Chen, Oliver Sch\"utze\thanks{Oliver Sch\"utze is with Department of Computer Science, CINVESTAV, Mexico City, Mexico} and J. Q. Sun}%

\markboth{IEEE Transactions on Industrial Informatics}{Shell \MakeLowercase{\textit{et al.}}: Bare Demo of IEEEtran.cls for Journals}

\maketitle

\begin{abstract}
This paper presents a data-driven framework for estimating the remaining useful life (RUL) of mechanical systems. Two major components make up the framework: a multi-layer perceptron as base regressor and an evolutionary computation algorithm for the tuning of data-related parameters. On the data side, the framework makes use of a strided time window along with a piecewise linear model to estimate the RUL label for each time window within the training sets. Tuning the data-related parameters using the optimization framework here presented allows for the use of simple regressor models, e.g. neural networks with few hidden layers and few neurons at each layer, which can in turn be deployed in environments with very limited resources such as embedded systems. The proposed method is evaluated on the publicly available CMAPS dataset. The accuracy of the proposed method is compared against other state-of-the art methods available in the literature and it is shown to perform better while making use of a simpler, compact model.
\end{abstract}


\begin{keywords}
artificial neural networks, moving time window, RUL estimation, CMAPS, prognostics, evolutionary algorithms
\end{keywords}

%\IEEEpeerreviewmaketitle
}}%
%BeginExpansion
%TCIDATA{LaTeXparent=0,0,RUL1.tex}

\title{A Neural Network-Evolutionary Computational Framework for Remaining Useful Life Estimation of Aircraft Engine}

\author{David Laredo\thanks{David Laredo, Zhaoyin Chen and J. Q. Sun are with Department of Mechanical Engineering, University of California, Merced, CA 95343.}, Zhaoyin Chen, Oliver Sch\"utze\thanks{Oliver Sch\"utze is with Department of Computer Science, CINVESTAV, Mexico City, Mexico} and J. Q. Sun}%

\markboth{IEEE Transactions on Industrial Informatics}{Shell \MakeLowercase{\textit{et al.}}: Bare Demo of IEEEtran.cls for Journals}

\maketitle

\begin{abstract}
This paper presents a data-driven framework for estimating the remaining useful life (RUL) of mechanical systems. Two major components make up the framework: a multi-layer perceptron as base regressor and an evolutionary computation algorithm for the tuning of data-related parameters. On the data side, the framework makes use of a strided time window along with a piecewise linear model to estimate the RUL label for each time window within the training sets. Tuning the data-related parameters using the optimization framework here presented allows for the use of simple regressor models, e.g. neural networks with few hidden layers and few neurons at each layer, which can in turn be deployed in environments with very limited resources such as embedded systems. The proposed method is evaluated on the publicly available CMAPS dataset. The accuracy of the proposed method is compared against other state-of-the art methods available in the literature and it is shown to perform better while making use of a simpler, compact model.
\end{abstract}


\begin{keywords}
artificial neural networks, moving time window, RUL estimation, CMAPS, prognostics, evolutionary algorithms
\end{keywords}

%\IEEEpeerreviewmaketitle

%EndExpansion


\section{Introduction}

\label{sec:rul_intro}

Traditionally, maintenance of mechanical systems has been carried out based on
scheduling strategies. Such strategies are often costly and less capable of
meeting the increasing demand of efficiency and reliability
\cite{Gebraeel2005, Zaidan2013}. Condition based maintenance (CBM) also known
as intelligent prognostics and health management (PHM) allows for maintenance
based on the current health of the system, thus cutting costs and increasing
the reliability of the system \cite{Zhao2017}. Here, we refer to prognostics
as the estimation of remaining useful life. The remaining useful life (RUL) of
a system can be estimated based on its historical data. This data-driven
approach can help optimize maintenance schedules to avoid engineering failures
and to save costs \cite{Lee2014}.

The existing PHM methods can be grouped into three different categories:
model-based \cite{Yu2001} , data-driven \cite{Liu2009, Mosallam2013} and
hybrid approaches \cite{Pecht2010, Liu2012}. Model-based approaches attempt to
incorporate physical models of the system into the estimation of the RUL. If
the system degradation is modeled precisely, model-based approaches usually
exhibit better performance than data-driven approaches \cite{Qian2017}. This
comes at the expense of having extensive a priori knowledge of the underlying
system and having a fine-grained model of the system, which can involve
expensive computations. On the other hand, data-driven approaches use pattern
recognition to detect changes in system states. Data-driven approaches are
appropriate when the understanding of the first principles of the system
operation is not comprehensive or when the system is sufficiently complex such
jet engines, car engines and complex machinery so that developing an accurate
model is prohibitively difficult.

Common disadvantages for the data-driven approaches are that they usually
exhibit wider confidence intervals than model-based approaches and that a fair
amount of data is required for training. Many data-driven algorithms have been
proposed. Good prognostics results have been achieved. Among the most popular
algorithms we can find artificial neural networks (ANNs) \cite{Gebraeel2004},
support vector machine (SVM) \cite{Benkedjouh2013}, Markov hidden chains (MHC)
\cite{Dong2007} and so on. Over the past few years, data-driven approaches
have gained more attention in the PHM community. A number of machine learning
techniques, especially neural networks, have been applied successfully to
estimate the RUL of diverse mechanical systems. ANNs have demonstrated good
performance in modeling highly nonlinear, complex, multi-dimensional systems
without any prior knowledge on the system behavior \cite{Li2018}. While the
confidence limits for the RUL predictions cannot be analytically provided
\cite{Sikorska2011}, the neural network approaches are promising for
prognostic problems.

Neural networks for estimating the RUL of jet engines has been previously
explored in \cite{Lim2016} where the authors propose a multi-layer perceptron
(MLP) coupled with a feature extraction (FE) method and a time window for the
generation of the features for the MLP. In the publication, the authors
demonstrate that a moving window combined with a suitable feature extractor
can improve the RUL prediction as compared with the studies with other similar
methods in the literature. In \cite{Li2018}, the authors explore a deep
learning ANN architecture, the so-called convolutional neural networks (CNNs),
where they demonstrate that by using a CNN without any pooling layers coupled
with a time window the predicted RUL is further improved.

In this paper we propose a novel framework for estimating the RUL of complex
mechanical systems. The framework consists of a MLP to estimate the RUL of the
system, coupled with an evolutionary algorithm for the fine tuning of
data-related parameters, i.e. parameters that define the shape and quality of
the features used by the MLP. The publicly available NASA CMAPS dataset
\cite{CMAPS2008} is used to assess the efficiency and reliability of the
proposed framework. This approach allows for simple and small MLP to obtain
better results than those reported in the current literature while using less
computing power.

The remainder of this paper is organized as follows. The CMAPS dataset is
presented in Section \ref{sec:rul_dataset}. The framework and its components
are thoroughly reviewed in Section \ref{sec:method}. The method is evaluated
using the CMAPS dataset in Section \ref{sec:rul_eval}. A comparison with the
state-of-the-art is also provided. Finally, the conclusions are presented in
Section \ref{sec:conclusions}.

\section{NASA C-MAPSS Dataset}

\label{sec:rul_dataset}

The NASA CMAPS dataset is used to evaluate performance of the proposed method
\cite{CMAPS2008}. The CMAPS dataset contains simulated data produced using a
model based simulation program developed by NASA. The dataset is further
divided into 4 subsets composed of multi-variate temporal data obtained from
21 sensors.

For each of the 4 subsets a training and a test set are provided. The training
sets include run-to-failure sensor records of multiple aero-engines collected
under different operational conditions and fault modes as described in Table
\ref{TabCMAPSS}.

The data is arranged in an $n\times26$ matrix where $n$ is the number of data
points in each subset. The first two variables represent the engine and cycle
numbers, respectively. The following three variables are operational settings
which correspond to the conditions in Table \ref{TabCMAPSS} and have a
substantial effect on the engine performance. The remaining variables
represent the 21 sensor readings that contain the information about the engine
degradation over time.

Each trajectory within the training and test sets represents the life cycles
of the engine. Each engine is simulated with different initial health
conditions, i.e. no initial faults. For each trajectory of an engine the last
data entry corresponds to the cycle at which the engine is found faulty. On
the other hand, the trajectories of the test sets terminate at some point
prior to failure, hence the need to predict the remaining useful life. The aim
of the MLP NN model is to predict the RUL of each engine in the test set. The
actual RUL values of test trajectories are also included in the dataset for
verification. Further discussions of the dataset and details on how the data
is generated can be found in \cite{Saxena2008}.

\subsection{Performance Evaluation}

\label{sec:rul_metrics}

To evaluate the performance of the proposed approach on the CMAPS dataset we
make use of two scoring indicators, namely the Root Mean Squared Error (RMSE)
denoted as $e_{rms}(d)$ and a score proposed in \cite{Saxena2008} which we
refer as the RUL Health Score (RHS) denoted as $s_{rh}(d)$. The two scores are
defined as follows,
\begin{equation}
e_{rms} = \sqrt{ \frac{1}{N} \sum_{i=1}^{N}{d_{i}^{2}}} \label{eq:rmse}%
\end{equation}
%

\begin{align}
s_{rh}  &  = \frac{1}{N} \sum_{i=1}^{N}{s_{i}}\nonumber\\
s_{i}  &  =
\begin{cases}
e^{-\frac{d_{i}}{13}} - 1, & d_{i} < 0\\
e^{\frac{d_{i}}{10}} - 1, & d_{i} \geq0,
\end{cases}
\label{eq:rhs}%
\end{align}
where $N$ is the total number of samples in the test set and $d = \hat{y} - y$
is the error between the estimated RUL values $\hat{y}$, and the actual RUL
values $y$ for each engine within the test set. It is important to notice that
$s_{rh}(d)$ penalizes late predictions more than early predictions since
usually late predictions lead to more severe consequences in fields such as aerospace.

\section{Framework Description}

\label{sec:method}

In this section, the proposed ANN-EA based method for prognostics is
presented. The model consists of a multi-layer perceptron (MLP) as the main
regressor for estimating the RUL of the engines in the CMAPS dataset. For the
training sets, the feature vectors are generated by using a moving time window
while a label vector is generated with the RUL of the engine. The label has a
constant RUL for the early cycles of the simulation, and becomes a linearly
decreasing function of the cycle in the remaining cycles. This is the
so-called piecewise linear degradation model \cite{Ramasso2014}. For the test
set, a time window is taken from the last sensor readings of the engine. The
data is used to predict the RUL of the engine.

The window-size $n_{w}$, window-stride $n_{s}$, and early-RUL $R_{e}$ are
data-related parameters, which for the sake of clarity and formalism in this
study form a vector $v \in\mathbb{Z}^{3}$ such that $v = (n_{w}, n_{s},
R_{e})$. The vector $v$ has a considerable impact on the quality of the
predictions by the regressor. It is computationally intensive to find the best
parameters of $v$ given the search space inherent to these parameters. In this
paper, we propose an evolutionary algorithm to fine tune the data-related
parameters $v$. The optimized parameter set $v$ allows the use of a simple
neural network architecture while attaining better results in terms of the
quality of the predictions compared with the results by other methods in the literature.

\subsection{The Neural Network Architecture}

After careful examinations of the CMAPS dataset, we propose to use a rather
simple MLP architecture for all the four subsets of the data. The
implementations are done in Python using the Keras/Tensorflow environment. The
source code is publicly available at the git repository \url{https://github.com/dlaredo/NASA_RUL_-CMAPS-}.

The choice of the network architecture is made by following an iterative
process: comparing 6 different architectures, training each for $100$
iterations using a mini-batch size of $512$ and averaging their results over
$10$ different runs. Two objectives are pursued during the iterations: 1) the
architecture must be minimal in terms of layers and neurons in each layer and
2) the performance indicators must be minimized.

The process for choosing the network architecture is as follows. First, choose
a $v$ for the experiment, say $v= (30, 1, 140)$. Next, six different ANN
architectures are defined, details of the architectures are provided in
Appendix \ref{sec:appendices}. For each of the six different architectures,
its performance is assessed using a cross-validation set from subset 1 of
CMAPS. Table \ref{table:tested_architectures_100} summarizes the results for
each tested architecture, while Table \ref{table:proposed_nn} presents the
architecture chosen for the remainder of this work. The chosen architecture
provides the best compromise between compactness and performance among the
tested architectures.

\subsection{Shaping the Data}

This section covers the data preprocessing applied to the raw sensor readings
in each of the datasets. Although the original datasets contain $21$ different
sensor readings, some of the sensors do not present much variance or convey
redundant information. These sensors are therefore discarded. In the end, only
$14$ sensor readings out of the $21$ are considered for this study. Their
indices are $\left\lbrace 2, 3, 4, 7, 8, 9, 11, 12, 13, 14, 15, 17, 20, 21
\right\rbrace $. The raw measurements are then used to create the strided time
windows with window size $n_{w}$ and window stride $n_{s}$. For the training
labels, $R_{e}$ is used at the early stages and then the RUL is linearly
decreased. The data is also normalized to be within the range $\left[  -1,1
\right]  $ using the min-max normalization.
\begin{equation}
\hat{x}_{i} = 2* \frac{x_{i} - min(x_{i})}{max(x_{i}) - min(x_{i})} - 1,
\label{eq:min_max_norm}%
\end{equation}
where $x_{i}$ denotes the $m$-dimensional vector whose components are all the
readings for the \textit{i-th} sensor and $\hat{x}_{i}$ is the normalized
$x_{i}$ vector.

\subsubsection{Time Window and Stride}

In multivariate time-series problems such as RUL, more information can be
generally obtained from the temporal sequence of the data as compared with the
multivariate data point at a single time stamp. For a time window of size
$n_{w}$ with a stride $n_{s}=1$, all the sensor readings in the time window
form a feature vector $\mathbf{x}$. This approach has successfully been tested
in \cite{Li2018,Lim2016} where the authors propose the use of a moving window
with sizes ranging from 20 to 30. In this paper, we propose not only the use of
a moving time window, but also a \textit{strided} time window that updates
more than one elements ($n_{s}>1$) at the time. A graphical depiction of the
strided time window is shown in Figure \ref{FigWindow}.

The use of a strided time window allows for the regressor to take advantage
not only of the previous information, but also to control the ratio at which
the algorithm is fed with new information. With the usual time window
approach, only one point is updated for every new time window. The strided
time window considered in this study allows for updating more than one points
at the time for the algorithm to make new information with less iterations. It
is believed that the information contained in the time window with stride size
$n_{s}>1$ is likely richer than the one contained in a time window with stride
size $n_{s}=1$.

\subsubsection{Piecewise Linear Degradation Model}

Different from common regression problems, the desired output value of the
input data is difficult to determine for a RUL problem. It is usually
impossible to evaluate the precise health condition and estimate the RUL of
the system at each time step without an accurate physics based model. For this
popular dataset, a piece-wise linear degradation model has been proposed in
\cite{Ramasso2014}. The model assumes that the engines have a constant RUL
label in the early cycles, and then the RUL starts degrading linearly until it
reaches 0 as shown in Figure \ref{FigRULinear}. The piecewise linear
degradation assumption is used in this work. We denote the value of the RUL in
the early cycles as $R_{e}$.

\subsection{Optimal Data Parameters}

\label{sec:otimal_data_params}

As mentioned in the previous sections the choice of the data-related
parameters $v$ has a large impact on the performance of the regressor. In this
section, we present the framework for picking the optimal combination of the
data-related parameters $n_{w}$, $n_{s}$ and $R_{e}$ while being
computationally efficient.

Recall that $v = (n_{w}, n_{s}, R_{e})$ specific to the CMAPS dataset are
bounded such that $n_{w} \in\left[  1, b\right]  $, $n_{s} \in\left[  1,
10\right]  $, and $R_{e} \in\left[  90, 140 \right]  $, where all the
variables are integer. The value of $b$ is different for different subsets of
the data, Table \ref{table:b_values} shows the different values of $b$ for
each subset.

Let $X(v)$ be the training/cross-validate/test sets parametrized by $v$ and
used by the MLP to perform the RUL estimation. Finally, let $f(v)=e_{rms}%
(X(v))$, recall from Equation (\ref{eq:rmse}) that $d = \hat{y} - y$ and that
$\hat{y}$ depends on $X(v)$. Note that one function evaluation of $f(v)$
implies training the MLP and computing the result of Equation (\ref{eq:rmse}).
Here we propose to fine tune $v$ as
\begin{equation}
\underset{v \in\mathbb{Z}^{3}}{\mathrm{min}} f(v)
\label{eq:optimization_problem}%
\end{equation}


The problem to find optimal data-related parameters has no analytical
descriptions. Therefore, no gradient information is available. An evolutionary
algorithm is the natural choice for this optimization problem.

\subsubsection{True Optimal Data Parameters}

The size of CMAPS dataset and the search space of $v$ allows for an exhaustive
search to be performed in order to find the true optimal data-related
parameters. We would like to emphasize that although exhaustive search is a
possibility for CMAPS dataset, it is in no way a possibility in a more general
setting. Nevertheless, the possibility to perform exhaustive search on the
CMAPS dataset can be exploited to demonstrate the accuracy of the chosen EA
and of the framework overall. In the following studies, we use the results and
computational efforts of the exhaustive search as benchmarks to examine the
accuracy and efficiency of the proposed approach.

We should note that the subsets of the data FD001 and FD003 have similar
features and that the subsets FD002 and FD004 have similar features. Because
of this, we have decided to just optimize the data-related parameters by
considering the subsets FD001 and FD002 only. An exhaustive search is
performed to find the true optimal values for $v$. As was the case when using
the DE method, the MLP is only trained for $20$ epochs. Table
\ref{table:true_optimal_data_params} shows the optimal as well as the worst
combinations of data-related parameters and the total number of function
evaluations used by the exhaustive search. It is important to notice that for 
this experiment the window size was limited to be larger than or equal to $15$, 
thus $n_w \leq 15$ for this experiment.

\subsubsection{Evolutionary Algorithm for Optimal Data Parameters}

\label{sec:ea_optimization_process}

Evolutionary algorithms (EAs) are a family of methods for optimization
problems. The methods do not make any assumptions about the problem, treating
it as a black box that merely provides a measure of quality given a candidate
solution. Furthermore, EAs do not require the gradient when searching for
optimal solutions, making them very suitable for applications such as neural networks.

For the current application, the differential evolution (DE) method is chosen
as the optimization algorithm \cite{Storn1997}. Though other meta-heuristic
algorithms may also be suitable for this application, the DE has been
stablished itself as one of the most reliable, robust and easy to use EAs.
Furthermore, a ready to use Python implementation is available through the
scipy package \cite{scipy}. Although the DE method does not have special
operators for treating integer variables, a very simple modification to the
algorithm, i.e. rounding every component of a candidate solution to its
nearest integer, is used for this work.

As mentioned earlier, evolutionary algorithms such as the DE use several
function evaluations when searching for the optimal solutions. Recall that for
this application, one function evaluation implies retraining the neural
network from scratch. This is not a desirable scenario, as obtaining the
optimal data-related parameters would entail an extensive computational
effort. Instead of running the DE for several iterations and with a large
population size, we propose to run it just for $30$ iterations, i.e. the
generations in the literature of evolutionary computation, with a population
size of $12$, which seems reasonable given the size of the search space of $v$.

During the optimization, the MLP is trained only $20$ epochs. The small number
of epochs of training the MLP is reasonable in this case because a small batch
of data is used in the training, which allows for fast convergence.
Furthermore, it is common to observe that the parameters leading to lower
score values in the early stages of the training are more likely to provide
better performance after more epochs of trainings. Details about the use of
the DE algorithm to find the optimal data-related parameters are listed in
Table \ref{table:de_hyperparams}.

The optimal data-related parameters for the subsets FD001 and FD002 found by
the DE algorithm are shown in Table \ref{table:optimal_data_params}. As can be
observed, the results are in fact very close to the true optimal ones listed
in Table \ref{table:true_optimal_data_params} for both the subsets of the
data. The computational burden is reduced by one order of magnitude when using
the DE method as compared to the exhaustive search for the true optimal
parameters. From the results in Table \ref{table:optimal_data_params}, it can
be observed that the maximum allowable time window is always preferred while,
on the other hand, small window strides yield better results. For the case of
early RUL, it can be observed that larger values of $R_{e}$ are favored.

\subsection{The ANN-EA RUL Estimation Framework}

Having described the major building blocks of the proposed method, we now
introduce the complete framework in Algorithm \ref{alg:rul_framework}.

\setcounter{algorithm}{0} \begin{algorithm}[H]
\caption{ANN-EA RUL estimation Framework}\label{alg:rul_framework}
\textbf{Input:} Initial set of data-related parameters $v \in \mathbb{Z}^n$, Raw training/testing data $X$ and training labels $y$\\
\textbf{Output:} Optimal set of data-related parameters $v^*$
\begin{algorithmic}[1]
\State Choose regressor architecture (ANN, SVM, linear/logistic regression, etc).
\State Define $f(v)$ as in Section\ref{sec:otimal_data_params}.
\State Optimize $f(v)$ using the preferred evolutionary algorithm, i.e. differential evolution, evolutionary strategies, genetic algorithm, etc, using the proposed guidelines from Section \ref{sec:ea_optimization_process}.
\State Use $v^*$ to train the regressor for as many epochs as needed.
\end{algorithmic}
\end{algorithm}


\section{Performance Evaluation of the Proposed Method}

\label{sec:rul_eval}

In this section we evaluate the performance of the proposed method. The
architecture of the MLP is described in Table \ref{table:proposed_nn} and will
be used throughout this section. The MLP was trained $10$ times for $200$
epochs each and tested in each subset of the CMAPS dataset.

For the first experiment, the combinations of window size $n_{w}$, window
stride $n_{s}$ and early RUL $R_{e}$ are presented in Table
\ref{table:data_params_de}.

The obtained results for $f(v)$ using the above setting are depicted in Table
\ref{table:results_ann_de}. Notice that the performances obtained for datasets
FD001 and FD002 are improved. This is due to the fact that the MLP is trained
for more epochs, thus obtaining better results.

Next, the possibility of using a single set of data-related parameters for all
the subsets is explored. For this experiment, the $n_{w}$ is fixed for all of
the four datasets, given that the maximum allowable window size for all
datasets is $18$. Hence, the data-related parameters are chosen as $v=(17, 1,
139)$. The results of predictions are shown in Table \ref{table:results_ann_1}.

As can be observed, the performance is decreased for the subsets FD001 and
FD003. This indicates that larger window sizes are beneficial for this
regression problem. Figures \ref{FigRMSEcomparison} and \ref{FigRHScomparison}
show a comparison of the scores for each dataset by changing the data-related
parameters.

\subsection{Comparison with Published Works}

In this section, the performance of the proposed method is compared against
other state-of-the-art methods. Most of the methods chosen to compare in this
section have only reported the results on the test set FD001 in terms of
$e_{rms}$. The results are shown in Table \ref{table:results_comparison}. The
$e_{rms}$ value of the proposed method in Table \ref{table:results_comparison}
is the mean value of 10 independent runs. The values of other methods are
identical to those reported in their respective original papers.

From the comparison studies of the prediction results, we can conclude that
the proposed method performs better than the majority of the chosen methods
when taking into consideration the whole dataset FD001. Two existing methods
come close to the performance of the proposed approach in this paper, namely
the time window ANN \cite{Lim2016} and the Networks Ensemble \cite{Zhang2016}.
While the performance of these two methods comes close to the results of the
proposed method in this paper, the proposed method is computationally more
efficient. Furthermore, the framework proposed in here is simple to understand
and implement, robust, generic and light-weight. These are the features that
are important to highlight when comparing the proposed method against other
state-of-the-art approaches.

\section{Conclusions}

\label{sec:conclusions}

We have presented a novel framework for predicting the RUL of mechanical
components. While the method was tested on the jet-engine specific dataset
CMAPS, the method is general enough that it can be applied to other similar
systems. The framework makes use of a strided moving time window to generate
the training and test sets. A shallow MLP to make the predictions of the RUL
has been found to be sufficient for the current dataset. The evolutionary
algorithm DE needs to be run just once to find the best data-related
parameters that optimize the scoring functions. The results presented in this
paper demonstrate that the proposed framework is accurate and computationally
efficient, which makes this framework suitable for applications that have
limited computational resources such as embedded systems. Furthermore, a
comparison with other state-of-the-art methods has shown that the proposed
method is the best overall performer.

Two major features of the proposed framework are its generality and
scalability. While for this study, very specific regressors and evolutionary
algorithms were chosen, many other combinations are possible and may be more
suitable for different applications. Furthermore, the framework can, in
principle, be used for model-construction, i.e. generating the best possible
neural network architecture tailored to a specific application.

\bibliographystyle{IEEEtran}
\bibliography{reference_rul_paper}
%

%TCIMACRO{\QSubDoc{Include IEEE Bio}{%TCIDATA{LaTeXparent=0,0,RUL1.tex}

\begin{biography}[{\includegraphics[width=1in,height=1.25in,clip]{BioPhotos/shufang.jpg}}]{David Laredo}
earned his Bachelor of Science from ....
\end{biography}

\begin{biography}[{\includegraphics[width=1in,height=1.25in,clip]{BioPhotos/kql.jpg}}]{Zhaoyin Chen}
is an undergraduate student in Computer Science Engineering at University of California, Merced.
\end{biography}

\begin{biography}[{\includegraphics[width=1in,height=1.25in,clip]{BioPhotos/kr.jpg}}]{Oliver Sch\"utze}
earned her Master of Science in Physical Therapy from Boston University in 1989 and her PhD in Biomechanics and Movement Sciences from the University of Delaware in 1998.  Dr. Rudolph joined the faculty of the Department of Physical Therapy and Program in Biomechanics and Movement Sciences at the University of Delaware in 1999 and is currently an Assistant Professor.  She has served as a manuscript reviewer for the Journal of Orthopaedic and Sports Physical Therapy, Archives of Physical Medicine and Rehabilitation, Journal of Applied Biomechanics and the European Journal of Applied Physiology and has reviewed grants for the National Institutes of Health Musculoskeletal and Rehabilitation Sciences Study section. Dr. Rudolph's research interests include analysis of walking in patients with knee osteoarthritis and neurological injury and development of research based physical therapy interventions.
\end{biography}


\begin{biography}[{\includegraphics[width=1in,height=1.25in,clip]{BioPhotos/jqs.jpg}}]{Jian-Qiao Sun}
earned a BS degree in Solid Mechanics from Huazhong University of Science and Technology in 1982, and PhD in Mechanical Engineering from University of California at Berkeley in 1988.  In 1994, Dr. Sun joined the faculty of the department of Mechanical Engineering at the University of Delaware as an Assistant Professor, was promoted to Associate Professor in 1998 and to Professor in 2003.  Dr. Sun joined the new campus of University of California at Merced in 2007 and is currently Professor and Chair of Mechanical Engineering. He serves as the Editor-in-Chief of the International Journal of Dynamics and Control by Springer.  His research interests include random vibrations, nonlinear controls, energy harvesting technologies, and data-driven modeling and analysis of complex mechanical systems.
\end{biography}


}}%
%BeginExpansion
%TCIDATA{LaTeXparent=0,0,RUL1.tex}

\begin{biography}[{\includegraphics[width=1in,height=1.25in,clip]{BioPhotos/shufang.jpg}}]{David Laredo}
earned his Bachelor of Science from ....
\end{biography}

\begin{biography}[{\includegraphics[width=1in,height=1.25in,clip]{BioPhotos/kql.jpg}}]{Zhaoyin Chen}
is an undergraduate student in Computer Science Engineering at University of California, Merced.
\end{biography}

\begin{biography}[{\includegraphics[width=1in,height=1.25in,clip]{BioPhotos/kr.jpg}}]{Oliver Sch\"utze}
earned her Master of Science in Physical Therapy from Boston University in 1989 and her PhD in Biomechanics and Movement Sciences from the University of Delaware in 1998.  Dr. Rudolph joined the faculty of the Department of Physical Therapy and Program in Biomechanics and Movement Sciences at the University of Delaware in 1999 and is currently an Assistant Professor.  She has served as a manuscript reviewer for the Journal of Orthopaedic and Sports Physical Therapy, Archives of Physical Medicine and Rehabilitation, Journal of Applied Biomechanics and the European Journal of Applied Physiology and has reviewed grants for the National Institutes of Health Musculoskeletal and Rehabilitation Sciences Study section. Dr. Rudolph's research interests include analysis of walking in patients with knee osteoarthritis and neurological injury and development of research based physical therapy interventions.
\end{biography}


\begin{biography}[{\includegraphics[width=1in,height=1.25in,clip]{BioPhotos/jqs.jpg}}]{Jian-Qiao Sun}
earned a BS degree in Solid Mechanics from Huazhong University of Science and Technology in 1982, and PhD in Mechanical Engineering from University of California at Berkeley in 1988.  In 1994, Dr. Sun joined the faculty of the department of Mechanical Engineering at the University of Delaware as an Assistant Professor, was promoted to Associate Professor in 1998 and to Professor in 2003.  Dr. Sun joined the new campus of University of California at Merced in 2007 and is currently Professor and Chair of Mechanical Engineering. He serves as the Editor-in-Chief of the International Journal of Dynamics and Control by Springer.  His research interests include random vibrations, nonlinear controls, energy harvesting technologies, and data-driven modeling and analysis of complex mechanical systems.
\end{biography}



%EndExpansion


\clearpage


\onecolumn%

%TCIMACRO{\TeXButton{Begin+Table+Center+Cap}{\begin{table}
%\begin{center}
%\caption{C-MAPSS Dataset details.}}}%
%BeginExpansion
\begin{table}
\begin{center}
\caption{C-MAPSS Dataset details.}%
%EndExpansion
%

\begin{tabular}
[c]{l|cccc}\hline
& \multicolumn{4}{c}{C-MAPSS}\\
Dataset & FD001 & FD002 & FD003 & FD004\\\hline\hline
Training Trajectories & 100 & 260 & 100 & 248\\
Test Trajectories & 100 & 259 & 100 & 248\\
Operating Conditions & 1 & 6 & 1 & 6\\
Fault Modes & 1 & 1 & 2 & 2\\\hline
\end{tabular}
\label{TabCMAPSS}%

%TCIMACRO{\TeXButton{End+Center+Table}{\end{center}
%\end{table}
%}}%
%BeginExpansion
\end{center}
\end{table}
%EndExpansion
%

%TCIMACRO{\TeXButton{Begin+Table+Center+Cap}{\begin{table}
%\begin{center}
%\caption{Results for different architectures for subset 1, 100 epochs.}}}%
%BeginExpansion
\begin{table}
\begin{center}
\caption{Results for different architectures for subset 1, 100 epochs.}%
%EndExpansion
%

\begin{tabular}
[c]{l|cccc|cccc}\hline
& \multicolumn{4}{|c}{RMSE} & \multicolumn{4}{|c}{RHS}\\
Tested Architecture & Min. & Max. & Avg. & STD & Min. & Max. & Avg. &
STD\\\hline\hline
Architecture 1 & 15.51 & 17.15 & 16.22 & 0.49 & 4.60 & 7.66 & 5.98 & 0.91\\
Architecture 2 & 15.24 & 16.46 & 15.87 & 0.47 & 4.07 & 6.26 & 5.29 & 0.82\\
Architecture 3 & 15.77 & 17.27 & 16.15 & 0.45 & 5.11 & 8.25 & 5.93 & 0.94\\
Architecture 4 & 15.13 & 17.01 & 15.97 & 0.47 & 3.90 & 7.54 & 5.65 & 1.2\\
Architecture 5 & 16.39 & 17.14 & 16.81 & 0.23 & 5.19 & 6.58 & 5.98 & 0.42\\
Architecture 6 & 16.42 & 17.36 & 16.87 & 0.30 & 5.15 & 7.09 & 6.12 &
0.62\\\hline
\end{tabular}
\label{table:tested_architectures_100}%

%TCIMACRO{\TeXButton{End+Center+Table}{\end{center}
%\end{table}}}%
%BeginExpansion
\end{center}
\end{table}%
%EndExpansion
%

%TCIMACRO{\TeXButton{Begin+Table+Center+Cap}{\begin{table}
%\begin{center}
%\caption{Proposed Neural Network architecture}.}}%
%BeginExpansion
\begin{table}
\begin{center}
\caption{Proposed Neural Network architecture}.%
%EndExpansion
%

\begin{tabular}
[c]{llll}\hline
Layer & Shape & Activation & Additional Information\\\hline\hline
Fully connected & \multicolumn{1}{c}{20} & \multicolumn{1}{c}{ReLU} &
$L1=0.1,L2=0.2$\\
Fully connected & \multicolumn{1}{c}{20} & \multicolumn{1}{c}{ReLU} &
$L1=0.1,L2=0.2$\\
Fully connected & \multicolumn{1}{c}{1} & \multicolumn{1}{c}{Linear} &
$L1=0.1,L2=0.2$\\\hline
\end{tabular}
\label{table:proposed_nn}%

%TCIMACRO{\TeXButton{End+Center+Table}{\end{center}
%\end{table}}}%
%BeginExpansion
\end{center}
\end{table}%
%EndExpansion
%

%TCIMACRO{\TeXButton{Begin+Table+Center+Cap}{\begin{table}
%\begin{center}
%\caption{Allowed values for $b$ per subset.}}}%
%BeginExpansion
\begin{table}
\begin{center}
\caption{Allowed values for $b$ per subset.}%
%EndExpansion
%

\begin{tabular}
[c]{c|cccc}\hline
& FD001 & FD002 & FD003 & FD004\\\hline
$b$ & 30 & 20 & 30 & 18\\\hline
\end{tabular}
\label{table:b_values}%

%TCIMACRO{\TeXButton{End+Center+Table}{\end{center}
%\end{table}}}%
%BeginExpansion
\end{center}
\end{table}%
%EndExpansion
%

%TCIMACRO{\TeXButton{Begin+Table+Center+Cap}{\begin{table}
%\begin{center}
%\caption{Exhaustive search results for subsets FD001 and F002.}}}%
%BeginExpansion
\begin{table}
\begin{center}
\caption{Exhaustive search results for subsets FD001 and F002.}%
%EndExpansion
%

\begin{tabular}
[c]{l|crcrr}\hline
Dataset & argmin $v$ & min $f(v)$ & argmax $v$ & max $f(v)$ & Function
evals.\\\hline\hline
FD001 & $\left[  24,1,127\right]  $ & \multicolumn{1}{c}{$15.11$} & $\left[
25,10,94\right]  $ & \multicolumn{1}{c}{$85.19$} & \multicolumn{1}{c}{8160}\\
FD002 & $\left[  16,1,138\right]  $ & \multicolumn{1}{c}{$30.93$} & $\left[
17,10,99\right]  $ & \multicolumn{1}{c}{$59.78$} & \multicolumn{1}{c}{3060}%
\\\hline
\end{tabular}
\label{table:true_optimal_data_params}%

%TCIMACRO{\TeXButton{End+Center+Table}{\end{center}
%\end{table}}}%
%BeginExpansion
\end{center}
\end{table}%
%EndExpansion


\bigskip%
%TCIMACRO{\TeXButton{Begin+Table+Center+Cap}{\begin{table}
%\begin{center}
%\caption{Differential evolution hyper-parameters.}}}%
%BeginExpansion
\begin{table}
\begin{center}
\caption{Differential evolution hyper-parameters.}%
%EndExpansion
%

\begin{tabular}
[c]{llll}\hline
Population Size & Generations & Strategy & MLP epochs\\\hline
\multicolumn{1}{c}{12} & \multicolumn{1}{c}{30} & \multicolumn{1}{c}{Best1Bin}
& \multicolumn{1}{c}{20}\\\hline
\end{tabular}
\label{table:de_hyperparams}%

%TCIMACRO{\TeXButton{End+Center+Table}{\end{center}
%\end{table}}}%
%BeginExpansion
\end{center}
\end{table}%
%EndExpansion
%

%TCIMACRO{\TeXButton{Begin+Table+Center+Cap}{\begin{table}
%\begin{center}
%\caption
%{Data-related parameters for each subset obtained with differential evolution.}%
%}}%
%BeginExpansion
\begin{table}
\begin{center}
\caption
{Data-related parameters for each subset obtained with differential evolution.}%
%EndExpansion
%

\begin{tabular}
[c]{l|crr}\hline
Dataset & argmin $v$ & min $f(v)$ & Function evals.\\\hline\hline
FD001 & $\left[  24,1,129\right]  $ & \multicolumn{1}{c}{$15.24$} &
\multicolumn{1}{c}{372}\\
FD002 & $\left[  17,1,139\right]  $ & \multicolumn{1}{c}{$30.95$} &
\multicolumn{1}{c}{372}\\\hline
\end{tabular}
\label{table:optimal_data_params}%

%TCIMACRO{\TeXButton{End+Center+Table}{\end{center}
%\end{table}}}%
%BeginExpansion
\end{center}
\end{table}%
%EndExpansion
%

%TCIMACRO{\TeXButton{Begin+Table+Center+Cap}{\begin{table}
%\begin{center}
%\caption{Data-related parameters for each subset as obtained by DE.}}}%
%BeginExpansion
\begin{table}
\begin{center}
\caption{Data-related parameters for each subset as obtained by DE.}%
%EndExpansion
%

\begin{tabular}
[c]{lrrr}\hline
Dataset & $n_{w}$ & $n_{s}$ & $R_{e}$\\\hline\hline
FD001 & 24 & 1 & 129\\
FD002 & 17 & 1 & 139\\
FD003 & 24 & 1 & 129\\
FD004 & 17 & 1 & 139\\\hline
\end{tabular}
\label{table:data_params_de}%

%TCIMACRO{\TeXButton{End+Center+Table}{\end{center}
%\end{table}}}%
%BeginExpansion
\end{center}
\end{table}%
%EndExpansion
%

%TCIMACRO{\TeXButton{Begin+Table+Center+Cap}{\begin{table}
%\begin{center}
%\caption
%{Scores for each dataset using the data-related parameters obtained by DE (Second architecture).}%
%}}%
%BeginExpansion
\begin{table}
\begin{center}
\caption
{Scores for each dataset using the data-related parameters obtained by DE (Second architecture).}%
%EndExpansion
%

\begin{tabular}
[c]{l|cccc|cccc}\hline
& \multicolumn{4}{|c}{RMSE} & \multicolumn{4}{|c}{RHS}\\
Data Subset & min & max & avg & STD & min & max & avg & STD\\\hline\hline
FD001 & 14.24 & 14.57 & 14.39 & 0.11 & 3.25 & 3.58 & 3.37 & 0.11\\
FD002 & 28.90 & 29.23 & 29.09 & 0.11 & 45.99 & 53.90 & 50.69 & 2.17\\
FD003 & 14.74 & 16.18 & 15.42 & 0.50 & 4.36 & 6.85 & 5.33 & 0.95\\
FD004 & 33.25 & 35.10 & 34.74 & 0.53 & 58.52 & 78.62 & 74.77 & 5.88\\\hline
\end{tabular}
\label{table:results_ann_de}%

%TCIMACRO{\TeXButton{End+Center+Table}{\end{center}
%\end{table}}}%
%BeginExpansion
\end{center}
\end{table}%
%EndExpansion
%

%TCIMACRO{\TeXButton{Begin+Table+Center+Cap}{\begin{table}
%\begin{center}
%\caption
%{Scores for each dataset using the single set of data-related parameters.}}}%
%BeginExpansion
\begin{table}
\begin{center}
\caption
{Scores for each dataset using the single set of data-related parameters.}%
%EndExpansion
%

\begin{tabular}
[c]{l|rrrr|rrrr}\hline
& \multicolumn{4}{|c}{RMSE} & \multicolumn{4}{|c}{RHS}\\
Data Subset & min & max & avg & STD & min & max & avg & STD\\\hline\hline
FD001 & 16.74 & 17.23 & 17.06 & 0.18 & 8.08 & 9.26 & 8.64 & 0.45\\
FD002 & 29.77 & 30.12 & 29.94 & 0.11 & 54.90 & 63.53 & 58.82 & 2.39\\
FD003 & 16.94 & 18.52 & 17.77 & 0.39 & 6.02 & 10.18 & 7.64 & 1.37\\
FD004 & 33.95 & 35.46 & 35.14 & 0.44 & 58.71 & 67.91 & 65.78 & 2.64\\\hline
\end{tabular}
\label{table:results_ann_1}%
%TCIMACRO{\TeXButton{End+Center+Table}{\end{center}
%\end{table}}}%
%BeginExpansion
\end{center}
\end{table}%
%EndExpansion
%

%TCIMACRO{\TeXButton{Begin+Table+Center+Cap}{\begin{table}
%\begin{center}
%\caption
%{Performance comparison of the proposed method and the latest related papers on the CMAPS dataset.}%
%}}%
%BeginExpansion
\begin{table}
\begin{center}
\caption
{Performance comparison of the proposed method and the latest related papers on the CMAPS dataset.}%
%EndExpansion
%

\begin{tabular}
[c]{l|r}\hline
Method & $e_{rms}$\\\hline
ESN trained by Kalman Filter \cite{Peng2012} & 63.45\\
Support Vector Machine Classifier \cite{Louen2013} & 29.82\\
Time Window Neural Network \cite{Lim2016} & 15.16\\
Multi-objective deep belief networks ensemble \cite{Zhang2016} & 15.04\\
Deep Convolutional Neural Network \cite{Babu2016} & 18.45\\
\textbf{Proposed method with $n_{w}=30$, $n_{s}=1$ and $R_{e}=128$} &
14.87\\\hline
\end{tabular}
\label{table:results_comparison}%

%TCIMACRO{\TeXButton{End+Center+Table}{\end{center}
%\end{table}}}%
%BeginExpansion
\end{center}
\end{table}%
%EndExpansion
%

%TCIMACRO{\TeXButton{Clearpage}{\clearpage}}%
%BeginExpansion
\clearpage
%EndExpansion
%

%TCIMACRO{\FRAME{ftbpFU}{4.0421in}{1.5212in}{0pt}{\Qcb{Graphical depiction of
%the time window used in this framework.}}{\Qlb{FigWindow}}{time_window.png}%
%{\special{ language "Scientific Word";  type "GRAPHIC";
%maintain-aspect-ratio TRUE;  display "USEDEF";  valid_file "F";
%width 4.0421in;  height 1.5212in;  depth 0pt;  original-width 10.933in;
%original-height 4.0672in;  cropleft "0";  croptop "1";  cropright "1";
%cropbottom "0";  filename 'Figures/time_window.png';file-properties "XNPEU";}}
%}%
%BeginExpansion
\begin{figure}
[ptb]
\begin{center}
\includegraphics[
%natheight=4.067200in,
%natwidth=10.933000in,
%height=1.5212in,
width=4.0421in
]%
{Figures/time_window.png}%
\caption{Graphical depiction of the time window used in this framework.}%
\label{FigWindow}%
\end{center}
\end{figure}
%EndExpansion


%

%TCIMACRO{\FRAME{ftbpFU}{3.2897in}{2.3661in}{0pt}{\Qcb{Piecewise linear
%degradation for RUL.}}{\Qlb{FigRULinear}}{test_engine.png}%
%{\special{ language "Scientific Word";  type "GRAPHIC";
%maintain-aspect-ratio TRUE;  display "USEDEF";  valid_file "F";
%width 3.2897in;  height 2.3661in;  depth 0pt;  original-width 5.3463in;
%original-height 3.8337in;  cropleft "0";  croptop "1";  cropright "1";
%cropbottom "0";  filename 'Figures/test_engine.png';file-properties "XNPEU";}}
%}%
%BeginExpansion
\begin{figure}
[ptb]
\begin{center}
\includegraphics[
natheight=3.833700in,
natwidth=5.346300in,
height=2.3661in,
width=3.2897in
]%
{Figures/test_engine.png}%
\caption{Piecewise linear degradation for RUL.}%
\label{FigRULinear}%
\end{center}
\end{figure}
%EndExpansion
%

%TCIMACRO{\FRAME{ftbpFU}{3.5405in}{2.2995in}{0pt}{\Qcb{Comparison of RMSE
%results for different sets of data-related parameters.}}%
%{\Qlb{FigRMSEcomparison}}{rmse_comparisson.png}%
%{\special{ language "Scientific Word";  type "GRAPHIC";
%maintain-aspect-ratio TRUE;  display "USEDEF";  valid_file "F";
%width 3.5405in;  height 2.2995in;  depth 0pt;  original-width 5.6593in;
%original-height 3.6599in;  cropleft "0";  croptop "1";  cropright "1";
%cropbottom "0";
%filename 'Figures/rmse_comparisson.png';file-properties "XNPEU";}} }%
%BeginExpansion
\begin{figure}
[ptb]
\begin{center}
\includegraphics[
natheight=3.659900in,
natwidth=5.659300in,
height=2.2995in,
width=3.5405in
]%
{Figures/rmse_comparisson.png}%
\caption{Comparison of RMSE results for different sets of data-related
parameters.}%
\label{FigRMSEcomparison}%
\end{center}
\end{figure}
%EndExpansion
%

%TCIMACRO{\FRAME{ftbpFU}{3.5405in}{2.3013in}{0pt}{\Qcb{Comparison of RHS
%results for different sets of data-related parameters.}}%
%{\Qlb{FigRHScomparison}}{rhs_comparisson.png}%
%{\special{ language "Scientific Word";  type "GRAPHIC";
%maintain-aspect-ratio TRUE;  display "USEDEF";  valid_file "F";
%width 3.5405in;  height 2.3013in;  depth 0pt;  original-width 5.6559in;
%original-height 3.6599in;  cropleft "0";  croptop "1";  cropright "1";
%cropbottom "0";
%filename 'Figures/rhs_comparisson.png';file-properties "XNPEU";}} }%
%BeginExpansion
\begin{figure}
[ptb]
\begin{center}
\includegraphics[
natheight=3.659900in,
natwidth=5.655900in,
height=2.3013in,
width=3.5405in
]%
{Figures/rhs_comparisson.png}%
\caption{Comparison of RHS results for different sets of data-related
parameters.}%
\label{FigRHScomparison}%
\end{center}
\end{figure}
%EndExpansion


\clearpage


\appendix


\section*{Tested Neural Network Architectures}

\label{sec:appendices}

\begin{table}[tbh]
\centering
\begin{tabular}
[c]{llll}\hline
Layer & Shape & Activation & Additional Information\\\hline\hline
Fully connected & 20 & ReLU & L1 = 0.1, L2 = 0.2\\
Fully connected & 20 & ReLU & L1 = 0.1, L2 = 0.2\\
Fully connected & 1 & Linear & L1 = 0.1, L2 = 0.2\\\hline
\end{tabular}
\caption{Proposed Neural Network architecture 1.}%
\label{table:proposed_nn_1}%
\end{table}

\begin{table}[tbh]
\centering
\begin{tabular}
[c]{llll}\hline
Layer & Shape & Activation & Additional Information\\\hline\hline
Fully connected & 50 & ReLU & L1 = 0.1, L2 = 0.2\\
Fully connected & 20 & ReLU & L1 = 0.1, L2 = 0.2\\
Fully connected & 1 & Linear & L1 = 0.1, L2 = 0.2\\\hline
\end{tabular}
\caption{Proposed Neural Network architecture 2.}%
\label{table:proposed_nn_2}%
\end{table}

\begin{table}[tbh]
\centering
\begin{tabular}
[c]{llll}\hline
Layer & Shape & Activation & Additional Information\\\hline
Fully connected & 100 & ReLU & L1 = 0.1, L2 = 0.2\\
Fully connected & 50 & ReLU & L1 = 0.1, L2 = 0.2\\
Fully connected & 1 & Linear & L1 = 0.1, L2 = 0.2\\\hline
\end{tabular}
\caption{Proposed Neural Network architecture 3.}%
\label{table:proposed_nn_3}%
\end{table}

\begin{table}[tbh]
\centering
\begin{tabular}
[c]{llll}\hline
Layer & Shape & Activation & Additional Information\\\hline
Fully connected & 250 & ReLU & L1 = 0.1, L2 = 0.2\\
Fully connected & 50 & ReLU & L1 = 0.1, L2 = 0.2\\
Fully connected & 1 & Linear & L1 = 0.1, L2 = 0.2\\\hline
\end{tabular}
\caption{Proposed Neural Network architecture 4.}%
\label{table:proposed_nn_4}%
\end{table}

\begin{table}[tbh]
\centering
\begin{tabular}
[c]{llll}\hline
Layer & Shape & Activation & Additional Information\\\hline\hline
Fully connected & 20 & ReLU & L1 = 0.1, L2 = 0.2\\
Fully connected & 1 & Linear & L1 = 0.1, L2 = 0.2\\\hline
\end{tabular}
\caption{Proposed Neural Network architecture 5.}%
\label{table:proposed_nn_5}%
\end{table}

\begin{table}[tbh]
\centering
\begin{tabular}
[c]{llll}\hline
Layer & Shape & Activation & Additional Information\\\hline\hline
Fully connected & 10 & ReLU & L1 = 0.1, L2 = 0.2\\
Fully connected & 1 & Linear & L1 = 0.1, L2 = 0.2\\\hline
\end{tabular}
\caption{Proposed Neural Network architecture 6.}%
\label{table:proposed_nn_6}%
\end{table}
\end{document}