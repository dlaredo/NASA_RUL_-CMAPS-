

%\runauthor{Xin, Qin and Sun}

\begin{frontmatter}

\title{A Neural Network-Evolutionary Computational Framework for Remaining Useful Life Estimation of Mechanical Systems}

\begin{comment}
\author{David Laredo$^{1}$, Zhaoyin Chen$^{1}$, Oliver Sch\"utze$^{2}$ and Jian-Qiao Sun$^{1}$}
\address{
$^{1}$Department of Mechanical Engineering\\
School of Engineering, University of California\\
Merced, CA 95343, USA\\
$^{2}$Department of Computer Science, CINVESTAV\\ 
Mexico City, Mexico\\
Corresponding author. Email: jqsun@ucmerced.edu}
\end{comment}

\begin{abstract}
This paper presents a framework for estimating the remaining useful life (RUL) of mechanical systems. The framework consists of a multi-layer perceptron and an evolutionary algorithm for optimizing the data-related parameters. The framework makes use of a strided time window along with a piecewise linear model to estimate the RUL for each mechanical component. Tuning the data-related parameters in the optimization framework allows for the use of simple models, e.g. neural networks with few hidden layers and few neurons at each layer, which may be deployed in environments with limited resources such as embedded systems. The proposed method is evaluated on the publicly available C-MAPSS dataset. The accuracy of the proposed method is compared against other state-of-the art methods in the literature. The proposed method is shown to perform better than the compared methods while making use of a compact model.

\end{abstract}


\begin{keyword}
artificial neural networks\sep
moving time window\sep
RUL estimation\sep
prognostics\sep
evolutionary algorithms
\end{keyword}

\end{frontmatter}
