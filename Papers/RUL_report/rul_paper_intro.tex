\section{Introduction}

Traditionally, maintenance of mechanical systems has been carried out based on scheduling strategies, nevertheless strategies such as breakdown corrective maintenance and scheduled preventive maintenance are often costly and less capable of meeting the increasing demand of efficiency and reliability \cite{gebraeel_2005, zaidan_2013}. Condition Based Maintenance (\gls{cbm}) also known as intelligent Prognostics and Health Management (\gls{pmh}) allows for maintenance based on the current health of the system, thus cutting costs and increasing the reliability of the system \cite{zhao_2017}. To avoid confusion, here we define prognostics as the estimation of remaining useful component life. The Remaining Useful Life (\gls{rul}) of a system can be estimated based on history trajectory data, this approach which we refer here as data-driven can help improve maintenance schedules to avoid engineering failures and save costs \cite{lee_2014}. This paper proposes a Machine Learning (\gls{ml}) approach for \gls{rul} estimation.

The existing \gls{pmh} methods can be grouped into three different categories: model-based approaches \cite{yu_2001} , data-driven approaches \cite{liu_2009, mosallam_2013} and hybrid approaches \cite{pecht_2010, liu_2012}.

Model-based approaches attempt to incorporate physical models of the system into the estimation of the \gls{rul}. If the system degradation is modeled  precisely, model-based approaches usually exhibit better performance than data-driven approaches \cite{qian_2017}, nevertheless this comes at the expense of having extensive a prior knowledge of the underlying system and having a fine-grained model of such system (which usually involve expensive computations). On the other hand data-driven approaches tend to use pattern recognition to detect changes in system states. Data-driven approaches are appropriate when the understanding of first principles of system operation is not comprehensive or when the system is sufficiently complex (i.e. jet engines, car engines, complex machinery) such that developing an accurate model is prohibitively expensive. Common disadvantages for the data-driven approaches are that they usually exhibit wider confidence intervals than model-based approaches and that a fair amount of data is required for training.