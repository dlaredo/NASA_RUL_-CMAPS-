\section{Performance evaluation}
\label{sec:rul_metrics}

The evaluate the performance of the proposed approach we make use of two scoring indicators, namely the Root Mean Squared Error (\gls{rmse}) and a score function proposed in \cite{Saxena2008} which we refer in this work as \gls{rul} Health Score (\gls{rhs}). 

\begin{equation}
RMSE = \sqrt{ \frac{1}{N} \sum_{i=1}^{N}{d_i^2}}
\end{equation}

\begin{align}
s &= \sum_{i=1}^{N}{s_i}\\
s_i &= \begin{cases} 
      e^{-\frac{d_i}{13}} - 1 & d_i < 0, \\
      e^{-\frac{d_i}{10}} - 1 & d_i \geq 0
\end{cases}
\end{align}

where $s$ denotes the score and $N$ is the total number of testing data samples. $d_i = RUL_i^p - RUL_i$, that is the error between the estimated \gls{rul} value and the actual \gls{rul} value for the \textbf{i-th} testing sample. The (\gls{rhs}) function penalizes late predictions more than early predictions since usually late predictions lead to more severe consequences in fields such as aerospace.

